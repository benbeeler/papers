\documentclass[review]{elsarticle}
\usepackage{lineno,hyperref}
\modulolinenumbers[5]
\usepackage[margin=1in]{geometry}
\usepackage{graphicx}
\usepackage{placeins}
\usepackage{comment}

\bibliographystyle{elsarticle-num}

\begin{document}
\begin{frontmatter}
\title{First principles study of fission product and fuel-cladding chemical interaction mitigating dopant energetics in $\alpha$ U}

\author[inl]{Benjamin Beeler\corref{qwe}}
\cortext[qwe]{Corresponding author}
\ead{benjamin.beeler@inl.gov}
\author[inl]{Chao Jiang}
\author[inl]{Yongfeng Zhang}
\address[inl]{Idaho National Laboratory, Idaho Falls, ID 83415}

\begin{abstract}

Uranium-zirconium (U-Zr) and uranium-plutonium-zirconium (U-Pu-Zr) alloy fuels have a history of usage in sodium-cooled fast reactors and have recently regained interest due to the possibility of incorporating minor actinides into the fuel in order to reduce the quantity of long-lived radioisotopes generated as nuclear waste. The periphery of the fuel is primarily comprised of the $\alpha$ phase of U. This presents a variety of unique challenges for fuel operation, largely due to the unique unique crystal structure of $\alpha$-U and the related anisotropic properties. A primary phenomenon of interest is early stage -U tearing, leading to porosity in the fuel which can have an impact on fuel performance and evolution. However, the extent of tearing is largely unknown and can vary over fuel compositions and irradiation conditions. In this work, molecular dynamics simulations are performed to analyze a variety of anisotropic properties in $\alpha$-U, in an attempt to provide information and insight into the phenomenon of $\alpha$-U tearing

\end{abstract}
\end{frontmatter}

\linenumbers

\section{Introduction}

Uranium-zirconium (U-Zr) and uranium-plutonium-zirconium (U-Pu-Zr) alloy fuels have a history of usage in sodium-cooled fast reactors. Not only does the U-Zr fuel (as well as U-Pu-Zr) generate a harder neutron spectrum as compared to traditional ceramic fuels, but it also offers excellent neutron economy and high burnup capability [1]. Recently, U-Zr fuels have regained interest due to the possibility of incorporating minor actinides into the fuel, and as such the metallic fuel alloys would serve to reduce the quantity of long-lived radioisotopes generated as nuclear waste [2]. 

A variety of microstructural phenomena affect the performance of U-Zr based alloy fuels, including constituent redistribution, significant swelling,  fission gas release, and fuel-clad chemical interaction. Each of these individual phenomena are affected by the complex compositional and phase variation present within the fuel. Typically, the periphery of the fuel is comprised of the phase of U (often coupled with $\delta$-UZr2), while the interior of the fuel pin exhibits the  phase, particularly during operation. The  phase presents a variety of unique challenges for fuel operation, the bulk of which relate to its unique crystal structure. The structure of  is face-centered orthorhombic and is shown in Fig. 1.

The unit cell contains four atoms and can be understood as alternating face-centered tetragonal planes (stacked in the z direction), as corrugated planes of atoms (normal to the y-direction), or as a distorted hexagonal close-packed structure. Any of these descriptions exemplifies the underlying anisotropy associated with the atomic configurations. Unsurprisingly, -U exhibits a variety of anisotropic properties. These include elastic constants [3], electrical thermal conductivity [4], threshold displacement energy [5] and thermal expansion [6]. Additionally, there are likely other unknown anisotropic properties that have either only been lightly explored, or untouched by scientific research. One important result of such an anisotropic structure is the phenomenon of U tearing. In reactor, the periphery of the fuel undergoes rapid tearing, generating elongated, textured voids. This is understood to be due to the phase undergoing anisotropic volume changes due to temperature gradients, irradiation, or both. The extent of tearing is largely unknown and can vary over fuel compositions and irradiation conditions. Increasing the fundamental understanding of how -U responds to temperature, irradiation, and mechanical stresses will provide a basis for predicting porosity formation early in fuel life, and thus improve the ability of fuel performance codes to accurately model the evolution of metallic fuel.

In this work, molecular dynamics simulations are performed to analyze a variety of anisotropic properties in . The thermal expansion, elastic constants and defect formation energies are studied as a function of temperature. The ents and surface energy are investigated as a function of both 2 temperature and crystallographic direction. Tensile testing is performed to obtain directional and temper-ature dependent Young's modulus, yield strength, and yield strain. Finally, irradiation-induced growth investigated for dirent point defect types.


\FloatBarrier

\section{Computational Details}

Molecular dynamics simulations are performed utilizing the LAMMPS [7] software package and the U- Mo ADP [8] interatomic potential. In order to determine the equilibrium volume at a given temperature, a supercell of 288 atoms (6x3x4 unit cells) in the structure was relaxed in an NPT ensemble, relaxing each x, y, and z component individually, with a damping parameter of 0.1. A Langevin thermostat in the Gronbech-Jensen-Farago [9, 10] formalism is utilized with the damping parameter set to 0.1 ps. The system is equilibrated for 100 ps, and the energy, volume, pressure and lattice constants are averaged over th l 25 ps of the simulation. This equilibration is performed for temperatures from 300 K up to 600 K in increments of 100 K, which spans both the temperature range of stability for U and potential temperatures observed on the periphery of the fuel pin. Elastic constants were calculated by performing displacements of the system and using the resultant changes in stress to compute the elastic stiess tensor, as provided by the LAMMPS distribution. Thermal fluctuations in these systems can lead to variations in the elastic constant calculations. In order to achieve tatistical certainty for each simulation of elastic constants, 100 samples are taken, with a sampling interval of 10 timesteps, as dened within the LAMMPS elastic constant scripts. The equilibrated system is input, a deformation is performed, and the system is equilibrated for 70 ps, averaging over the nal 20 ps in order to obtain the subsequent changes in the stress tensor. This simulation produces the nine elastic constants, or the complete stress tensor. Two unique simulations are performed for each temperature in order to further assure statistical signi cance.

Further anisotropic elastic properties are investigated by examining a system of 3600 atoms, equilibrated from 300 K to 600 K and applying a given strain. Three systems are generated with an aspect ratio of approximately 3:1, for each of the individual x, y and z directions. Convergence testing is performed with regard to system size and the strain rate in order to assure results are not dependent upon the speci c simulation setup. The systems are initially equilibrated in an NPT ensemble for 100 ps with a timestep of 2 fs. Subsequently, systems are deformed, utilizing an engineering strain rate of 505 per second. The strain is applied along the elongated direction of the supercell. The systems are actively equilibrated under strain in an NVT ensemble. The individual components of the stress tensor are tracked as a function of time, allowing for the determination of stress-strain curves. Systems are evolved to a maximum strain of 0.1(10%).

In order to calculate point defect energies and dion, a supercell of 2304 atoms (12x6x8 unit cells) is utilized. Ten unique simulations are performed for each of the defect-free, interstitial, and vacancy systems in order to ensure statistical signicance. Systems are equilibrated for 100 ps, averaging the energy over the nal 25 ps. Subsequently, the system is allowed to evolve for 10 ns and the mean-squared displacement (msd or hr2i) is tracked as a function of time. The msd for a defect-free system is subtracted from the msd for a system with a defect to ensure that ects of lattice thermal motion are removed from the investigationof point defect dision. The dision cient is extracted from the Einstein equation (D=hr2i/6t). The msd for each individual direction (x, y and z) is output in order to determine potential anisotropies in point defect dusion. The self-dision cient is then determined by equation 1Dself = Dintcint + Dvaccvac (1) where Dint and Dvac are the defect-speci c dision coients and cint and cvac are the interstitial and vacancy concentrations taken from exp(Edef =kT), where Edef is either the interstitial or vacancy formation energy averaged from 400 K to 600 K. The entropic contribution on the defect concentration is assumed to be small and thus treated as a factor of one. There does not currently exist data on the entropic contributions of point defects in U.

Surface energy is calculated from equation 2


where E the energy the system with two surfaces, E is the energy per atom of the perfect crystal
of SA is the total surface area (two surfaces are present in the system) and N is the number of
atoms in the system with two surfaces. Ten unique simulations are performed for each system to ensure
statistical signnce of the results. Three different surfaces are analyzed, the (100), (010) and (001) to
determine if anisotropy exists with regards to surface energy. The systems consist of approximately 4000
atoms (depending on the surface plane being analyzed) and are equilibrated for 100 ps, averaging data over
the final 25 ps to extract surface energies.

Irradiation-induced growth is investigated by equilibrating a 30x16x20 unit cell (38400 atoms) at 500 K
in an NPT ensemble for 100 ps, and subsequently periodically injecting interstitials, generating vacancies,
or creating Frenkel pairs. A defect is introduced into the system every 10 ps, while equilibration is on-going.
A total of 500 defects are inserted into the system over 5 ns leading to a maximum defect concentration
of 1.3%, which is excessively high in order to exacerbate potential volumetric changes. The system is then
equilibrated for another 15 ns to allow for defect evolution and reorganization. For the insertion of Frenkel
pairs, the insertion rate is kept constant (every 10 ps), but a total number of 2000 Frenkel pairs are inserted
into the system. This is due to the ability of point defects to recombine during the simulation. As such, no
additional relaxation time is included. The volume of the system and the individual supercell dimensions
are tracked to investigate irradiation-induced growth and its anisotropy.

\FloatBarrier
\section{Results}
\subsection{$\alpha$-U Equilibrium Properties}

The equilibrium properties of U from 300 K to 600 K are shown in Table 1, including energy per atom, volume per atom, individual lattice constants and heat capacity. The experimental lattice constants at 300 K were determined to be 2.854 , 5.868 and 4.955 for x (a), y (b) and z (c), respectively [11]. The MD results in Table 1 compare very favorably to the experimental results, with only a slightly overestimation of the y and z lattice constants. The experimental values for heat capacity for U from 300 K to 600 K from Konings [12] range from 27.7 J/mol-K up to 34.7 J/mol-K. Thus, this interatomic potential slightly under-predicts the experimental heat capacity, but is within 15% for all values and correctly predicts the general trends. Table 1: Calculated equilibrium properties of U.

E/at V/at x y z Cp
300 -4.201 20.716 2.850 5.818 4.997 25.6
400 -4.186 20.740 2.852 5.815 5.002 26.3
500 -4.171 20.776 2.855 5.811 5.009 27.3
600 -4.153 20.832 2.862 5.804 5.016 29.7

The evolution of the lattice constants is known to be anisotropic in U [6], and thus, the thermal
expansion in each individual a, b and c direction is unique. The comparison of MD-calculated changes in
lattice constants as a function of temperature to the experimental results is shown in Fig. 2. The trends in
lattice constant evolution for this work is consistent with the experimental results, in that there is expansion
in the a and c directions that are comparable, while there is contraction in the b direction. The degree of
expansion in the a and c directions is underestimated compared to experiment, while the degree of contraction
in the b direction is overestimated. However, there is excellent qualitative agreement, especially considering
the complexity and anisotropicity of the crystal structure.

\FloatBarrier

\section{Elastic Response of $\alpha$-U}

The elastic constants at 300 K are shown in Table 2, compared to results from density functional theory
(DFT) [13] and from experiment [3]. The elastic constants from this interatomic potential match fairly well
the experimental values, in a relative sense if not in an absolute sense. However, the bulk modulus, C12
and C33 are overestimated, while the C44, C55 and C66 elastic constants are underestimated. Interestingly,
the calculations from the potential more closely match experiments than do the DFT calculations. The
normalized elastic constants (with respect to their values at 300 K) are shown in Fig. 3 as a function of
temperature. There is general softening of the elastic constants with increasing temperature from 300 K
to 600 K as would be expected, but the degree of softening depends upon the individual elastic constants.
Analyzing the the bulk modulus provides an estimate for general behavior, and the bulk modulus softens by
13% (19 GPa). The maximum relative softening observed is 52% and occurs for the C55 elastic constant,
while the minimum relative softening is 5% and occurs for the C12 elastic constant. However, the maximum
magnitude in softening is for the C33 elastic constant, which softens by 54 GPa (19%). Thus, it is seen that
not only is the thermal expansion of the U structure anisotropic (Fig. 2), but the individual components
of the U elastic tensor vary anisotropically with temperature. This complex behavior of anisotropic elastic constants varying anisotropically has not been attempted to be implemented in any mesoscale or continuum scale fuel evolution models.

MD Expt [3] DFT [13]
C11 233 215 299
C22 203 199 231
C33 285 267 364
C12 97 46 59
C13 88 22 30
C23 115 108 144
C44 78 124 132
C55 34 73 150
C66 38 74 100
B 147 115 151

To more fully explore the anisotropic elastic response of U, stress-strain curves are developed in each
of the individual x, y and z directions from 300 K to 600 K. The stress-strain curves for each individual
direction at 300 K are shown in Fig. 4. It is observed that there exists a unique behavior for each direction.
There are three distinct yield strengths, yield strains and slopes to the stress-strain curves. It appears that
the y direction can obtain the greatest amount of elongation, while the x direction can elongate the least.
The z direction has the highest yield strength and the x direction has the lowest. This is vastly different
elastic response depending on direction.

The individual elastic response as a function of temperature is also analyzed, with the stress-strain curves
for the x direction from 300 K to 600 K are shown in Fig. 5. As expected, there is general softening with
increasing temperature. The yield strength and yield strain both decrease in magnitude with increasing
tempeature. The yield strength is reduced by a factor of three in the x direction. This analysis was
performed for all directions across the entire temperature range. The results are shown in Fig. 6 and are
tabulated in the appendix in tables tables A1 to A3. There is softening in all three directions with increasing
temperature, but not the same degree of softening. For example the yield strength, at 300 K the z direction
is the strongest, but the z directions softens more dramatically than the y direction and as such the y
direction is stronger at 600 K. Similarly, the x direction has an intermediate Young's modulus (comparing
8
directions) at 300 K, but presents the lowest Young's modulus at 600 K. However, the general relationship
among directions is retained across the temperature regime for the yield strain. This does in fact show that
in addition to the inherent elastic anisotropicity of the different directions, there exists an additional layer
of anisotropicity as a function of temperature for each direction for the different elastic responses presented
here. Figure

\FloatBarrier

\section{Point Defects}

Point defect formation energies as a function of temperature are shown in Fig. 7 for single vacancies and interstitials in U. Error bars included denote plus/minus one standard error of the mean. It is observed that point defect formation energies are generally consistent across this temperature regime, with the interstitial
formation energy approximately 2.5 eV and the vacancy formation energy is approximately 1.3 eV. Diffusion coefficients for each point defect were then determined by tracking the mean-squared displacement as a function of time. The results for diffusion coefficients are shown in Fig. 8, with results at 300 K excluded
due to a lack of diffusion on limited MD timescales. It is seen that interstitials diffuse slightly faster than
vacancies and posses a slightly lower migration energy (slope). An exponential t can be applied to the data
in Fig. 8 to extract a pre-factor and a migration barrier, which are shown in table 3. The migration barriers
compare very favorably to previous density functional theory studies in U showing migration barriers for
interstitials and vacancies of 0.2 eV and 0.34 eV, respectively [14].

Table 3: Point defect diffusion pre-factor (D0) and migration barrier (Em) for interstitials and vacancies.
D0 (m2/s) Em (eV)
Dint 2.40E-7 0.30
Dvac 6.56E-7 0.38
These individual point defect diffusion coefficients can be broken down into their directional constituents
(x, y and z ), and this has been done in Fig. 9, where r2 = x2 + y2 + z2 . For interstitials, it can be seen
that the z direction displays the most rapid diffusion, although it is only slightly faster than diffusion in
the y direction. Interstitial diffusion in the x direction is approximately 5X slower than in either of the two
directions over the given temperature range. For vacancies, diffusion occurs equally rapidly in the x and z
directions, which are both one order or magnitude faster than diffusion in the y direction. This additionally
agrees with a previous DFT study showing a larger migration barrier in the y direction, and equal barriers in
the x and z directions for vacancies in U. This work conrms previous assumptions regarding anisotropic
diffsion in U, while providing distinct diffusion coefficients for isolated point defects in each of the primary
directions. These deconstructed diffusion coefficients can be utilized in higher length scale rate theory or phase-field models to investigate complex evolution effects.


\section{Surface Energy}

The surface energy as a function of temperature for each of the individual x, y and z directions are shown in Fig. 10. Both the x and z directions exhibit the same surface energy of approximately 1.15 J/m2, while the y direction possesses a higher surface energy of approximately 1.27 J/m2. For all three directions, there is a slight increase in the surface energy as a function of increasing temperature. Although there is clear directional anisotropicity of the surface energy, there does not seem to be anisotropicity with respect to temperature, in that the relative magnitudes of the individual directions remain constant with respect to one another with increasing temperature. The effect of such anisotropic surface energies can affect preferential surface formation and orientation, void faceting, grain orientation and defect segregation. In addition, it is known that when -UZr is quenched to low temperatures, it decomposes to an U/UZr2 lamellar structure, whose morphology must in part be affcted by the interfacial energies (related to surface
energy) of the individual lamellae.


\section{Irradiation Surface Growth}

The irradiation induced volumetric changes due to insertion of vacancies are shown in Fig. 11. A vertical
dashed lined is overlaid at 5 ps where defect insertion was halted. The volume (V) and well as the x, y
and z directions are shown in Fig. 11, normalized to their values at time t=0. The volume is reduced by
approximately 0.7% with the insertion of 500 (1.3% composition) vacancies. The majority of the volume
change occurs in the x direction, with a secondary amount in the [ y] directly but very little change in the z
direction. Additionally, after defect insertion is stopped and the system is allowed to evolve, with vacancies diffusing and forming clusters, the volume change in the x and z directions remains constant, while the volume change in the y direction continues to decrease.
The irradiation induced volumetric changes due to insertion of interstitials are shown in Fig. 12. Similar
to Fig. 11, a vertical dashed lined is overlaid at 5 ps where defect insertion was halted and normalized V and
x, y and z directions are shown in Fig. 12. The volume increases by approximately 1.7% for a concentration
of 1.3% interstitials. The majority of the expansion takes place in the x direction, with a secondary amount
in the z direction and very minimal expansion in the y. After defect insertion is stopped and the system is
allowed to evolve, interstitials diffuse, cluster and reorient yielding further increase in dimension in the x,
while the volume remains constant. Thus, the expansion in the x is accompanied by reductions in length in the y and z directions.

The irradiation induced volumetric changes due to insertion of Frenkel pairs (vacancy-interstitial pair)
are shown in Fig. 13. It should be emphasized that a greater number of defects were inserted into the system
as Frenkel pairs (2000 vacancies AND 2000 interstitials) than from the simulations in Fig. 11 and Fig. 12
(500 vacancies OR 500 interstitials) but active defect annihilation was ongoing throughout the simulation.
As expected from the magnitudes of the volumetric changes in Fig. 11 and Fig. 12, the system tends to
swell and primarily in the x direction, although lesser increases in volume are observed for both the y and z
directions. Although a constant rate of defect insertion is applied in the system, the change in volume is very
jagged, this is due to the active annihilation of points defects throughout the simulation. It is found that
at the end of the simulation, there are approximately 130 Frenkel pairs, hence the much lower volumetric
changes in Fig. 13 compared to both Fig. 11 and Fig. 12.
An experimental result from Paine and Kittel [15] in 1958 showed single crystal uranium undergoing
irradiation growth positively in the y direction and negatively in the x direction, with no changes in the z
direction. This is in direct conflict with the results from this work, which showed that U likely undergoes
irradiation growth with expansion in the x direction, which is in line with the findings from Fig. 6 that the x direction is the softest of the three primary directions. It is possible that additional effects of surfaces, large scale clustering, fission products, or other heterogeneous effects are yielding experimental results that are

\section{Conclusions}
In this work, molecular dynamics simulations were performed to analyze a variety of anisotropic properties
in U. The anisotropic thermal expansion observed from experiments is confirmed, with expansion in the
x and z directions and contraction in the y direction with increasing temperature. The softening of the
elastic constants is also observed to be anisotropic with the C55 component softening by 52% from 300 K
to 600 K while the bulk modulus softens by 13%, or 19 GPa. The elastic response is investigated, showing
that U is the most stiff and the strongest in the z direction at low temperature, while it can elongate
the most in the y direction. The x exhibits both the lowest yield strength and yield strain of the three
primary directions. Point defect diffusion is also studied, showing interstitial diffusion slightly faster than
vacancy diffusion, comparing well with DFT results. The individual directional components of the diffusion coeffidient are deconstructed to show that interstitial diffusion occurs primarily in the y and z directions, while vacancy diffusion occurs primarily in the x and z directions. The surface energies were calculated which
showed that the y surface energy is significantly higher than either the x and z surfaces. Finally, irradiation-
induced growth is investigated for different point defect types, indicating a defect-induced expansion in the
x direction.

This work can be utilized to parametrize anisotropic elasticity, diffusion and interfacial energy models
in phase eld and continuum fuel performance codes. This work also brings forth the question of why the
experimental results for irradiation growth differ with the computational results. Further examinations
including a number of heterogeneous microstructural features on an atomistic and meso-scale are warranted.

\section{Acknowledgement}

Add NEAMS ackowledgement This manuscript has been authored by Battelle Energy Alliance, LLC with
the U.S. Department of Energy. The publisher, by accepting the article for publication, acknowledges that
the U.S. Government retains a nonexclusive, paid-up, irrevocable, worldwide license to publish or reproduce
the published form of this manuscript, or allow others to do so, for U.S. Government purposes. This research
made use of the resources of the High Performance Computing Center at Idaho National Laboratory, which
is supported by the Office of Nuclear Energy of the U.S. Department of Energy and the Nuclear Science
User Facilities.

\FloatBarrier

%\Appendix

Table A1: Young's Modulus of U.
X Y Z
300 165.8 144.1 191.1
400 160.3 138.5 181.8
500 139.3 120.5 181.8
600 109.5 119.4 175.5

Table A2: Yield Strength of  U.
X Y Z
300 5.8 9.7 10.3
400 4.8 8.9 8.8
500 3.7 8.1 7.3
600 2.4 7.1 5.8

Table A3: Yield Strain of U.
X Y Z
300 0.041 0.093 0.063
400 0.035 0.090 0.060
500 0.028 0.091 0.053
600 0.022 0.080 0.043


\bibliography{../MARMOTbib}


\end{document}  
