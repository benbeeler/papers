

\documentclass[utf8]{frontiersSCNS} % for Science, Engineering and Humanities and Social Sciences articles

\usepackage{url,hyperref,lineno,microtype,subcaption}
\usepackage[onehalfspacing]{setspace}
\usepackage{placeins}

\usepackage{xcolor}
\usepackage{siunitx}

\linenumbers


\def\keyFont{\fontsize{8}{11}\helveticabold }
\def\firstAuthorLast{Beeler {et~al.}} 
\def\Authors{Benjamin Beeler\,$^{1,2,*}$, Khadija Mahbuba\,$^{1}$, Yuhao Wang\,$^{3}$, and Andrea Jokisaari\,$^{2}$}

\def\Address{$^{1}$North Carolina State University, Raleigh, NC 27607 \\
$^{2}$Idaho National Laboratory, Idaho Falls, ID 83415\\
$^{3}$University of Michigan, Ann Arbor, MI 48109 }
\def\corrAuthor{2500 Stinson Dr, Raleigh, NC, 27607}
\def\corrEmail{bwbeeler@ncsu.edu}



\begin{document}
\onecolumn
\firstpage{1}

\title[AIMD calculation of temperature-dependent properties of $\alpha$-U]{Determination of thermal expansion, defect formation energy, and defect-induced strain of $\alpha$-U via \textit{ab initio} molecular dynamics} 

\author[\firstAuthorLast ]{\Authors} %This field will be automatically populated
\address{} %This field will be automatically populated
\correspondance{} %This field will be automatically populated

\extraAuth{}% If there are more than 1 corresponding author, comment this line and uncomment the next one.



\section{Supplementary Information}

This appendix includes tabulated data which has been presented and discussed above. 

\setcounter{table}{0}
\renewcommand{\thetable}{A\arabic{table}}

\begin{table}[h]
\caption{The calculated lattice parameter of $\alpha$-U from 100 K to 800 K.} \label{tab:lat}
\begin{center}
\begin{tabular}{|c|c|c|c|c|}
	\hline
	Temperature (K) & a$_0$ (\AA) & b$_0$ (\AA) & c$_0$ (\AA) & V/at (\AA$^3$) \\
	 \hline
100 &	2.72	& 5.96	& 4.91 &	79.75 \\
200 &	2.74 &	5.94 &	4.92 &	80.22  \\
300 &	2.78	& 5.91 &	4.93	& 80.83 \\
400 &	2.80 &	5.88 &	4.94	& 81.27 \\
500 &	2.82 &	5.86 &	4.95	& 81.67 \\
600 &	2.82 &	5.85 &	4.96	& 81.95 \\
700 &	2.83 &	5.84 &	4.97	 & 82.33 \\
800 &	2.84 &	5.83 &	4.98	& 82.63 \\
	 \hline
\end{tabular}
\end{center}
\label{default}
\end{table}

\begin{table}[h]
\caption{The calculated constant pressure heat capacity of $\alpha$-U from 100 K to 800 K, compared to experimental values from \cite{konings2010}. Units in J/mol-K.} \label{tab:cp}
\begin{center}
\begin{tabular}{|c|c|c|}
	\hline
	Temperature (K) & AIMD & Reference \\
	 \hline
150 &	27.7	& -	 \\
250 &	29.5 & 26.7  \\
350 &	30.5	& 28.7 \\
450 &	30.7 & 30.8  \\
550 &	32.5 & 33.2  \\
650 &	34.7 & 36.2  \\
750 &	35.1 & 39.8  \\
	 \hline
\end{tabular}
\end{center}
\label{default}
\end{table}

\begin{table}[h]
\caption{The interstitial and vacancy formation energy in $\alpha$-U as a function of temperature. AIMD results are compared to 0 K DFT data (denoted with *) from \cite{wirth2011}. Units in eV.} \label{tab:eform}
\begin{center}
\begin{tabular}{|c|c|c|}
	\hline
	Temperature (K) & Interstitial & Vacancy \\
	 \hline
0* & 4.42 & 1.69 \\
100 &	3.81 &	1.69 \\
200 &	3.82 &	1.56 \\
300 &	3.35 &	1.32 \\
400 &	3.13 &	1.16 \\
500 &	3.11 &	1.41 \\
600 &	3.14 &	1.23 \\
700 &	3.30 & 1.42 \\
800 &	3.58 &	1.93 \\
	 \hline
\end{tabular}
\end{center}
\label{default}
\end{table}

\begin{table}[h]
\caption{The defect-induced strain from interstitials and vacancies as a function of temperature. The strain is defined as $\Delta L/L_0$, where $L_0$ is the reference length at the given temperature. } \label{tab:strain}
\begin{center}
\begin{tabular}{|c|c|c|c|c|c|c|}
	\hline
	& \multicolumn{3}{c|}{Interstitial} & \multicolumn{3}{c|}{Vacancy} \\
	 \hline
Temperature (K)  &	a	& b	& c & a & b & c \\
\hline
100	&0.028	&-0.009	&0.002&		0.000	&-0.002	&0.000 \\
200	&0.020	&-0.006	&0.002&		0.001	&-0.002	&0.000 \\
300	&0.015	&-0.004	&0.003&		-0.003	&-0.001	&0.000 \\
400	&0.008	&0.001	&0.002&		-0.003	&0.000	&0.000 \\
500	&0.008	&0.000	&0.003&		-0.002	&-0.001	&0.000 \\
600	&0.005	&0.004	&0.002&		-0.001	&-0.002	&0.000 \\
700	&0.007	&0.000	&0.003&		0.000	&-0.003	&0.000 \\
800	&0.007	&0.000	&0.004&		0.001	&-0.004	&0.001 \\
	 \hline
\end{tabular}
\end{center}
\label{default}
\end{table}

\FloatBarrier

\bibliographystyle{frontiersinSCNS_ENG_HUMS} 
\bibliography{beelerbib}



\end{document}
