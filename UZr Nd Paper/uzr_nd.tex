\documentclass[review]{elsarticle}
\usepackage{lineno,hyperref}
\modulolinenumbers[5]
\usepackage[margin=1in]{geometry}
\usepackage{graphicx}
\usepackage{placeins}
\usepackage{comment}

\bibliographystyle{elsarticle-num}

\begin{document}
\begin{frontmatter}
\title{First principles study of fission product and fuel-cladding chemical interaction mitigating dopant energetics in $\alpha$ U}

\author[inl]{Benjamin Beeler\corref{qwe}}
\cortext[qwe]{Corresponding author}
\ead{benjamin.beeler@inl.gov}
\author[inl]{Chao Jiang}
\author[inl]{Yongfeng Zhang}
\address[inl]{Idaho National Laboratory, Idaho Falls, ID 83415}

\begin{abstract}

\end{abstract}
\end{frontmatter}

\linenumbers

\section{Introduction}
Uranium-zirconium (U-Zr) and uranium-plutonium-zirconium (U-Pu-Zr) alloy fuels have a history of usage in sodium-cooled fast reactors. Not only does the U-Zr fuel (as well as U-Pu-Zr) generate a harder neutron spectrum as compared to traditional ceramic fuels, but it also offers excellent neutron economy and high burnup capability \cite{hofman1997}. Recently, U-Zr fuels have regained interest due to the possibility of incorporating minor actinides into the fuel, and as such the metallic fuel alloys would serve to reduce the quantity of long-lived radioisotopes present in nuclear waste \cite{capriotti2017}. 

U-Zr alloys are typically employed as a series of fuel pellets within a given fuel rod, similar to UO$_{2}$ fuels. Unlike UO$_{2}$ fuels, dramatic swelling is inevitable, and is typically accounted for by manufacturing fuels with a smear density of approximately 75{\%}. This allows for approximately 30\% swelling, which is sufficient to allow fission gas release \cite{beck1968}. The nature of swelling is anisotropic in these fuels, largely due to the difference in swelling behavior between the hotter center of the fuel and the colder periphery \cite{hofman1990}. During operation, the phenomenon of constituent redistribution takes place. This results in three distinct radial regions within the fuel. The innermost region is Zr-rich, the intermediate region is Zr-poor and the outermost region has a nominal Zr concentration. This phenomenon is due to both the effect of the temperature gradient on phase equilibria and the diffusion of species along the temperature gradient. This concentration variance as a function of radius in combination with the temperature gradient leads to the $\gamma$ phase being present in the interior of the fuel pellet, while the $\alpha$ phase predominates the periphery \cite{kobayashi1990, kim2004}. The radially anisotropic swelling within the fuel pellet is due to the variation in phases as a function of radius.

Despite these unique fuel evolution phenomena, typically the life-limiting fuel behavior is fuel-cladding chemical interaction (FCCI). FCCI can be viewed in two respects, the first is the chemical attack of fission products on the cladding, degrading the cladding integrity and increasing the likelihood of cladding failure. The second aspect of FCCI is cladding, particularly iron (Fe), diffusing into the fuel and forming low-melting intermetallic structures. This second aspect of FCCI occurs once the fuel has swelled and come into contact with the cladding, while the first aspect can occur as long as fission product transport to the cladding has occurred. 

Fission product transport occurs through a variety of mechanisms, and the actual combination of dominant diffusion pathways is not well understood. Likely, fission products migrate through the fuel via bulk diffusion, surface diffusion along bubble surfaces, and liquid diffusion through the sodium (Na) coolant. All of these mechanisms need to be understood and described in order to have a complete picture of fission product transport as a function of burnup in the fuel, such that accurate description and prediction of FCCI rates can be produced. 

One means of delaying or inhibiting FCCI is through inclusion of dopant species during fuel fabrication. Dopants can exist in the UZr matrix or form intermetallics with U or Zr, and serve as sinks for fission products, either delaying or inhibiting their bulk diffusion. Dopants can be applied to both UZr and UPuZr fuels. A series of experimental studies on potential dopants has been performed analyzing the microstructure of fuels with dopant and fission product inclusions in an attempt to predict the viability of various dopant species \cite{all of benson papers}

In this work, the $\alpha$ phase of U is analyzed to model $\alpha$-UZr. Various dopant and fission products species are investigated, analyzing their formation energy as isolated point defects, as well as their affinity for binding to vacancies, Zr, and each other.



\section{Computational Details}
Systems are investigated using the Vienna $\textit{ab initio}$ Simulation Package (VASP) \cite{vasp1, vasp2, vasp3, vasp4}.  The projector augmented wave (PAW) method \cite{paw1, paw2} is utilized within density functional theory \cite{dft1, dft2}.  Calculations are performed using the Perdew-Burke-Ernzerhof (PBE) \cite{pbe1, pbe2} generalized gradient approximation (GGA) for the description of the exchange-correlation.  Methfessel and Paxton's smearing method \cite{methfessel} of the first order is used with a width of 0.1 eV to determine partial occupancies for each wavefunction.  Wavefunction optimization was truncated when the energy difference was less than 10$^{-5}$ eV.  The optimization procedure was truncated when the residual forces for the relaxed atoms were less than 0.01 eV/{\AA}.  Symmetry is switched off for all calculations.  A support grid is utilized for the evaluation of augmentation charges.  Non-spherical contributions from the gradient corrections inside the PAW spheres are included.  Convergence testing was performed for this system in the model of Huang and Wirth \cite{wirth2011} (utilized Perdew-Wang (PW91) \cite{pw91, pw91b} GGA).   The energy cutoff is increased, the supercell is increased in size, and the k-points mesh is refined until the energy of the pure system and the formation energy of a vacancy vary by less than 5 meV.  The convergence testing led to the choice of a 252 atom supercell (7x3x3 unit cells), a cutoff energy of 400 eV and a gamma-centered k-points mesh of 4x2x4, resulting in 20 k-points in the irreducible part of the Brillouin zone. The calculated lattice parameters are $\it{a}$ = 2.803 {\AA}, $\it{b}$ = 5.836 {\AA}, $\it{c}$ = 4.906 {\AA} and $\it{y}$ = 0.098.  These results compare very favorable to previous results \cite{wirth2011, soderlind2002, taylor2008}  and experiments \cite{barrett1963}.  

In order to obtain migration barriers for the various pathways investigated, we utilized the climbing-image nudged elastic band (CI-NEB) method \cite{neb1}.  Both single and multiple image implementations of the CI-NEB method were implemented to ensure that both the correct migration pathway and the correct migration barrier magnitude were obtained. 

\section{Results}

\subsection{Self-defect formation energies in $\alpha$U-(Zr)}

To verify the methodologies presented in this study, the formation energy of a single vacancy and a single interstitial in $\alpha$U were calculated utilizing equation \ref{eqn:vac} and \ref{eqn:int}, respectively,

\begin{equation}
\label{eqn:vac}
E_{f}^{vac} = E_{vac} - (n-1)*E_{pure}
\end{equation} 

\begin{equation}
\label{eqn:int}
E_{f}^{int} = E_{int} - (n+1)*E_{pure}
\end{equation} 

where E$_{vac}$ is the total energy of the system with a vacancy, E$_{int}$ is the total energy of the system with an interstitial, E$_{pure}$ is the energy per atom of a defect-free system, and \textit{n} is the number of atoms in the defect-free system. The vacancy formation energy is calculated to be 1.70 eV, which compares very favorably with the calculations of Huang and Wirth \cite{wirth2011} (1.69 eV).  

Nd interstitial 7.8 eV.  A Nd substitutional of 2.106 eV and U interstitial of 4.31 eV \cite{wirth2012} have a lower combined energy than the Nd interstitial.  Thus it would be expected that if a Nd interstitial is created, it would transform into a substitutional Nd atom and create a U SIA.  Therefore, only vacancy-mediated diffusion of Nd atoms is investigated.  Also, vacancy migration in the corrugated plane in four time as likely (migration barriers of 1.24 eV out of corrugated plane, 0.36 eV and 0.34 eV in corrugated plane \cite{wirth2011}, thus only vacancy migration in the corrugated plane is investigated.  

Pure U defects

\begin{table}[h!]
\caption{Formation energies for self-defects in alpha U.}
\label{tab:Eforms}
\begin{center}
\begin{tabular}{|c|c|c|}
     \hline
      Defect & E$_{form}$ & E$_{bind}$ \\
     \hline
     vacancy & 1.67 & - \\
     interstitial & 4.20 & - \\
     divacancy X & 3.81 & 0.47 \\
     divacancy Z & 3.62 & 0.27 \\
     divacancy Y & 3.30 & -0.05 \\
       \hline
\end{tabular}
\end{center}
\label{default}
\end{table}%

Zr defects

\begin{table}[h!]
\caption{Formation energies for Zr and Zr-vacancy defects in alpha U.}
\label{tab:Eforms}
\begin{center}
\begin{tabular}{|c|c|c|}
     \hline
      Defect & E$_{form}$ & E$_{bind}$ \\
     \hline
     Zr sub & 0.81 & - \\
     Zr int & 5.60 & - \\
     Zr Vac X & 2.30 & -0.18 \\
     Zr Vac Z & 1.97 & -0.52 \\
     Zr Vac Y & 2.54 & 0.05 \\
       \hline
\end{tabular}
\end{center}
\label{default}
\end{table}%

\FloatBarrier

\subsection{Fission product and dopant defect formation energies in $\alpha$U}

\begin{table}[h!]
\caption{Formation energies for Nd and Nd-vacancy defects in alpha U.}
\label{tab:Eforms}
\begin{center}
\begin{tabular}{|c|c|c|}
     \hline
      Defect & E$_{form}$ & E$_{bind}$ \\
     \hline
     Nd sub & -0.16 & - \\
     Nd int & 3.43 & - \\
     Nd Vac X & 1.55 & 0.04 \\
     Nd Vac Z & 1.46 & -0.05 \\
     Nd Vac Y & 1.47 & -0.04 \\
       \hline
\end{tabular}
\end{center}
\label{default}
\end{table}%

\begin{table}[h!]
\caption{Formation energies for Pd and Pd-vacancy defects in alpha U.}
\label{tab:Eforms}
\begin{center}
\begin{tabular}{|c|c|c|}
     \hline
      Defect & E$_{form}$ & E$_{bind}$ \\
     \hline
     Pd sub & 0.19 & - \\
     Pd int & 3.26 & - \\
     Pd Vac X & 1.67 & -0.20 \\
     Pd Vac Z & 1.59 & -0.27 \\
     Pd Vac Y & 1.82 & -0.05 \\
       \hline
\end{tabular}
\end{center}
\label{default}
\end{table}%


\begin{table}[h!]
\caption{Formation energies for Sn and Sn-vacancy defects in alpha U.}
\label{tab:Eforms}
\begin{center}
\begin{tabular}{|c|c|c|}
     \hline
      Defect & E$_{form}$ & E$_{bind}$ \\
     \hline
     Sn sub & 0.70 & - \\
     Sn int & 4.66 & - \\
     Sn Vac X & 2.01 & -0.36 \\
     Sn Vac Z & 1.55 & -0.82 \\
     Sn Vac Y & 2.27 & -0.11 \\
       \hline
\end{tabular}
\end{center}
\label{default}
\end{table}%


\begin{table}[h!]
\caption{Formation energies for Ce and Ce-vacancy defects in alpha U.}
\label{tab:Eforms}
\begin{center}
\begin{tabular}{|c|c|c|}
     \hline
      Defect & E$_{form}$ & E$_{bind}$ \\
     \hline
     Ce sub & 1.25 & - \\
     Ce int & - & - \\
     Ce Vac X & 1.90 & -1.03 \\
     Ce Vac Z & 1.75 & -1.18 \\
     Ce Vac Y & 2.09 & -0.84 \\
       \hline
\end{tabular}
\end{center}
\label{default}
\end{table}%

\FloatBarrier

\subsection{Formation energies of combined point defects of, and binding energies between, fission products, dopants and Zr in $\alpha$U}

\begin{table}[h!]
\caption{Formation energies for Pd-based defect complexes in alpha U.}
\label{tab:Eforms}
\begin{center}
\begin{tabular}{|c|c|c|}
     \hline
      Defect & E$_{form}$ & E$_{bind}$ \\
     \hline
     Nd-Pd X & 0.03 & 0.00 \\
     Nd-Pd Z & 0.00 & -0.02 \\
     Nd-Pd Y & 0.01 & -0.02 \\
     Ce-Pd X & 0.60 & -0.49 \\
     Ce-Pd Z & 0.45 & -0.64 \\
     Ce-Pd Y & 0.61 & -0.48 \\
     Zr-Pd X & 0.74 & -0.26 \\
     Zr-Pd Z & 0.47 & -0.53\\
     Zr-Pd Y & 0.74 & -0.27 \\     
           \hline
\end{tabular}
\end{center}
\label{default}
\end{table}%

\begin{table}[h!]
\caption{Formation energies for Pd-based defect complexes in alpha U.}
\label{tab:Eforms}
\begin{center}
\begin{tabular}{|c|c|c|}
     \hline
      Defect & E$_{form}$ & E$_{bind}$ \\
     \hline
     Nd-Sn X & 0.44 & -0.09 \\
     Nd-Sn Z & 0.46 & -0.07 \\
     Nd-Sn Y & 0.53 & -0.01 \\
     Ce-Sn X & - & - \\
     Ce-Sn Z & - & - \\
     Ce-Sn Y & - & - \\
     Zr-Sn X & 1.31 & -0.20 \\
     Zr-Sn Z & 0.94 & -0.57 \\
     Zr-Sn Y & 1.19 & -0.32 \\     
           \hline
\end{tabular}
\end{center}
\label{default}
\end{table}%


\FloatBarrier
\section{Conclusions}


\section{Acknowledgement}


\FloatBarrier

\bibliography{MARMOTbib}


\end{document}  
